% Options for packages loaded elsewhere
\PassOptionsToPackage{unicode}{hyperref}
\PassOptionsToPackage{hyphens}{url}
\PassOptionsToPackage{dvipsnames,svgnames,x11names}{xcolor}
%
\documentclass[
  letterpaper,
  DIV=11,
  numbers=noendperiod]{scrartcl}

\usepackage{amsmath,amssymb}
\usepackage{lmodern}
\usepackage{iftex}
\ifPDFTeX
  \usepackage[T1]{fontenc}
  \usepackage[utf8]{inputenc}
  \usepackage{textcomp} % provide euro and other symbols
\else % if luatex or xetex
  \usepackage{unicode-math}
  \defaultfontfeatures{Scale=MatchLowercase}
  \defaultfontfeatures[\rmfamily]{Ligatures=TeX,Scale=1}
\fi
% Use upquote if available, for straight quotes in verbatim environments
\IfFileExists{upquote.sty}{\usepackage{upquote}}{}
\IfFileExists{microtype.sty}{% use microtype if available
  \usepackage[]{microtype}
  \UseMicrotypeSet[protrusion]{basicmath} % disable protrusion for tt fonts
}{}
\makeatletter
\@ifundefined{KOMAClassName}{% if non-KOMA class
  \IfFileExists{parskip.sty}{%
    \usepackage{parskip}
  }{% else
    \setlength{\parindent}{0pt}
    \setlength{\parskip}{6pt plus 2pt minus 1pt}}
}{% if KOMA class
  \KOMAoptions{parskip=half}}
\makeatother
\usepackage{xcolor}
\setlength{\emergencystretch}{3em} % prevent overfull lines
\setcounter{secnumdepth}{5}
% Make \paragraph and \subparagraph free-standing
\ifx\paragraph\undefined\else
  \let\oldparagraph\paragraph
  \renewcommand{\paragraph}[1]{\oldparagraph{#1}\mbox{}}
\fi
\ifx\subparagraph\undefined\else
  \let\oldsubparagraph\subparagraph
  \renewcommand{\subparagraph}[1]{\oldsubparagraph{#1}\mbox{}}
\fi

\usepackage{color}
\usepackage{fancyvrb}
\newcommand{\VerbBar}{|}
\newcommand{\VERB}{\Verb[commandchars=\\\{\}]}
\DefineVerbatimEnvironment{Highlighting}{Verbatim}{commandchars=\\\{\}}
% Add ',fontsize=\small' for more characters per line
\usepackage{framed}
\definecolor{shadecolor}{RGB}{241,243,245}
\newenvironment{Shaded}{\begin{snugshade}}{\end{snugshade}}
\newcommand{\AlertTok}[1]{\textcolor[rgb]{0.68,0.00,0.00}{#1}}
\newcommand{\AnnotationTok}[1]{\textcolor[rgb]{0.37,0.37,0.37}{#1}}
\newcommand{\AttributeTok}[1]{\textcolor[rgb]{0.40,0.45,0.13}{#1}}
\newcommand{\BaseNTok}[1]{\textcolor[rgb]{0.68,0.00,0.00}{#1}}
\newcommand{\BuiltInTok}[1]{\textcolor[rgb]{0.00,0.23,0.31}{#1}}
\newcommand{\CharTok}[1]{\textcolor[rgb]{0.13,0.47,0.30}{#1}}
\newcommand{\CommentTok}[1]{\textcolor[rgb]{0.37,0.37,0.37}{#1}}
\newcommand{\CommentVarTok}[1]{\textcolor[rgb]{0.37,0.37,0.37}{\textit{#1}}}
\newcommand{\ConstantTok}[1]{\textcolor[rgb]{0.56,0.35,0.01}{#1}}
\newcommand{\ControlFlowTok}[1]{\textcolor[rgb]{0.00,0.23,0.31}{#1}}
\newcommand{\DataTypeTok}[1]{\textcolor[rgb]{0.68,0.00,0.00}{#1}}
\newcommand{\DecValTok}[1]{\textcolor[rgb]{0.68,0.00,0.00}{#1}}
\newcommand{\DocumentationTok}[1]{\textcolor[rgb]{0.37,0.37,0.37}{\textit{#1}}}
\newcommand{\ErrorTok}[1]{\textcolor[rgb]{0.68,0.00,0.00}{#1}}
\newcommand{\ExtensionTok}[1]{\textcolor[rgb]{0.00,0.23,0.31}{#1}}
\newcommand{\FloatTok}[1]{\textcolor[rgb]{0.68,0.00,0.00}{#1}}
\newcommand{\FunctionTok}[1]{\textcolor[rgb]{0.28,0.35,0.67}{#1}}
\newcommand{\ImportTok}[1]{\textcolor[rgb]{0.00,0.46,0.62}{#1}}
\newcommand{\InformationTok}[1]{\textcolor[rgb]{0.37,0.37,0.37}{#1}}
\newcommand{\KeywordTok}[1]{\textcolor[rgb]{0.00,0.23,0.31}{#1}}
\newcommand{\NormalTok}[1]{\textcolor[rgb]{0.00,0.23,0.31}{#1}}
\newcommand{\OperatorTok}[1]{\textcolor[rgb]{0.37,0.37,0.37}{#1}}
\newcommand{\OtherTok}[1]{\textcolor[rgb]{0.00,0.23,0.31}{#1}}
\newcommand{\PreprocessorTok}[1]{\textcolor[rgb]{0.68,0.00,0.00}{#1}}
\newcommand{\RegionMarkerTok}[1]{\textcolor[rgb]{0.00,0.23,0.31}{#1}}
\newcommand{\SpecialCharTok}[1]{\textcolor[rgb]{0.37,0.37,0.37}{#1}}
\newcommand{\SpecialStringTok}[1]{\textcolor[rgb]{0.13,0.47,0.30}{#1}}
\newcommand{\StringTok}[1]{\textcolor[rgb]{0.13,0.47,0.30}{#1}}
\newcommand{\VariableTok}[1]{\textcolor[rgb]{0.07,0.07,0.07}{#1}}
\newcommand{\VerbatimStringTok}[1]{\textcolor[rgb]{0.13,0.47,0.30}{#1}}
\newcommand{\WarningTok}[1]{\textcolor[rgb]{0.37,0.37,0.37}{\textit{#1}}}

\providecommand{\tightlist}{%
  \setlength{\itemsep}{0pt}\setlength{\parskip}{0pt}}\usepackage{longtable,booktabs,array}
\usepackage{calc} % for calculating minipage widths
% Correct order of tables after \paragraph or \subparagraph
\usepackage{etoolbox}
\makeatletter
\patchcmd\longtable{\par}{\if@noskipsec\mbox{}\fi\par}{}{}
\makeatother
% Allow footnotes in longtable head/foot
\IfFileExists{footnotehyper.sty}{\usepackage{footnotehyper}}{\usepackage{footnote}}
\makesavenoteenv{longtable}
\usepackage{graphicx}
\makeatletter
\def\maxwidth{\ifdim\Gin@nat@width>\linewidth\linewidth\else\Gin@nat@width\fi}
\def\maxheight{\ifdim\Gin@nat@height>\textheight\textheight\else\Gin@nat@height\fi}
\makeatother
% Scale images if necessary, so that they will not overflow the page
% margins by default, and it is still possible to overwrite the defaults
% using explicit options in \includegraphics[width, height, ...]{}
\setkeys{Gin}{width=\maxwidth,height=\maxheight,keepaspectratio}
% Set default figure placement to htbp
\makeatletter
\def\fps@figure{htbp}
\makeatother

\KOMAoption{captions}{tableheading}
\makeatletter
\@ifpackageloaded{tcolorbox}{}{\usepackage[many]{tcolorbox}}
\@ifpackageloaded{fontawesome5}{}{\usepackage{fontawesome5}}
\definecolor{quarto-callout-color}{HTML}{909090}
\definecolor{quarto-callout-note-color}{HTML}{0758E5}
\definecolor{quarto-callout-important-color}{HTML}{CC1914}
\definecolor{quarto-callout-warning-color}{HTML}{EB9113}
\definecolor{quarto-callout-tip-color}{HTML}{00A047}
\definecolor{quarto-callout-caution-color}{HTML}{FC5300}
\definecolor{quarto-callout-color-frame}{HTML}{acacac}
\definecolor{quarto-callout-note-color-frame}{HTML}{4582ec}
\definecolor{quarto-callout-important-color-frame}{HTML}{d9534f}
\definecolor{quarto-callout-warning-color-frame}{HTML}{f0ad4e}
\definecolor{quarto-callout-tip-color-frame}{HTML}{02b875}
\definecolor{quarto-callout-caution-color-frame}{HTML}{fd7e14}
\makeatother
\makeatletter
\makeatother
\makeatletter
\makeatother
\makeatletter
\@ifpackageloaded{caption}{}{\usepackage{caption}}
\AtBeginDocument{%
\ifdefined\contentsname
  \renewcommand*\contentsname{Contenidos}
\else
  \newcommand\contentsname{Contenidos}
\fi
\ifdefined\listfigurename
  \renewcommand*\listfigurename{Listado de Figuras}
\else
  \newcommand\listfigurename{Listado de Figuras}
\fi
\ifdefined\listtablename
  \renewcommand*\listtablename{Listado de Tablas}
\else
  \newcommand\listtablename{Listado de Tablas}
\fi
\ifdefined\figurename
  \renewcommand*\figurename{Figura}
\else
  \newcommand\figurename{Figura}
\fi
\ifdefined\tablename
  \renewcommand*\tablename{Tabla}
\else
  \newcommand\tablename{Tabla}
\fi
}
\@ifpackageloaded{float}{}{\usepackage{float}}
\floatstyle{ruled}
\@ifundefined{c@chapter}{\newfloat{codelisting}{h}{lop}}{\newfloat{codelisting}{h}{lop}[chapter]}
\floatname{codelisting}{Listado}
\newcommand*\listoflistings{\listof{codelisting}{Listado de Listados}}
\makeatother
\makeatletter
\@ifpackageloaded{caption}{}{\usepackage{caption}}
\@ifpackageloaded{subcaption}{}{\usepackage{subcaption}}
\makeatother
\makeatletter
\@ifpackageloaded{tcolorbox}{}{\usepackage[many]{tcolorbox}}
\makeatother
\makeatletter
\@ifundefined{shadecolor}{\definecolor{shadecolor}{rgb}{.97, .97, .97}}
\makeatother
\makeatletter
\makeatother
\makeatletter
\@ifpackageloaded{fontawesome5}{}{\usepackage{fontawesome5}}
\makeatother
\ifLuaTeX
\usepackage[bidi=basic]{babel}
\else
\usepackage[bidi=default]{babel}
\fi
\babelprovide[main,import]{spanish}
% get rid of language-specific shorthands (see #6817):
\let\LanguageShortHands\languageshorthands
\def\languageshorthands#1{}
\ifLuaTeX
  \usepackage{selnolig}  % disable illegal ligatures
\fi
\IfFileExists{bookmark.sty}{\usepackage{bookmark}}{\usepackage{hyperref}}
\IfFileExists{xurl.sty}{\usepackage{xurl}}{} % add URL line breaks if available
\urlstyle{same} % disable monospaced font for URLs
\hypersetup{
  pdftitle={Ecuaciones en Quarto con el paquete amsmath},
  pdfauthor={Eva María Mazcuñán Navarro},
  pdflang={es},
  colorlinks=true,
  linkcolor={blue},
  filecolor={Maroon},
  citecolor={Blue},
  urlcolor={Blue},
  pdfcreator={LaTeX via pandoc}}

\title{Ecuaciones en Quarto con el paquete \texttt{amsmath}}
\author{Eva María Mazcuñán Navarro}
\date{}

\begin{document}
\maketitle
\ifdefined\Shaded\renewenvironment{Shaded}{\begin{tcolorbox}[enhanced, interior hidden, borderline west={3pt}{0pt}{shadecolor}, breakable, boxrule=0pt, sharp corners, frame hidden]}{\end{tcolorbox}}\fi

\renewcommand*\contentsname{Contenidos}
{
\hypersetup{linkcolor=}
\setcounter{tocdepth}{3}
\tableofcontents
}
\hypertarget{introducciuxf3n}{%
\section{Introducción}\label{introducciuxf3n}}

El paquete \texttt{amsmath} añade nuevas funcionalidades a las
herramientas estandar de LaTeX para trabajar con ecuaciones. Permite en
particular:

\begin{itemize}
\tightlist
\item
  Dividir ecuaciones largas en varias líneas. Por ejemplo: \[
    \begin{split}
    P(x) & = 1  + x + x^2 \\
        & \hspace{2em} + x^3 + x^4 + x^5 \\
        & \hspace{4em} + x^6 + x^7 + x^8
    \end{split}
    \]
\item
  Agrupar varias ecuaciones en una determinada disposición. Por ejemplo
  para escribir un sistema de ecuaciones lineales como el siguiente: \[
    \left\{
    \begin{alignedat}{2} 
    2x & + {}   &  3y  & =  10 \\ 
    x & - {}   &   y  & =  0 
    \end{alignedat}
    \right.
    \]
\end{itemize}

En este artículo se relacionan las funcionalidades del paquete
\texttt{amsmath} que pueden usarse en un documento de Quarto
(\texttt{.qmd}) y funcionan tanto para el formato PDF como para el
formato HTML, permitiendo en ambos casos numerar las ecuaciones o grupos
de ecuaciones y crear referencias cruzadas.

\hypertarget{generalidades}{%
\section{Generalidades}\label{generalidades}}

Como aspectos comunes a las funcionalidades que aprenderemos en las
siguientes secciones, y en analogía con la sintaxis de los entornos
\texttt{array} y \texttt{tabular}, tenemos:

\begin{itemize}
\tightlist
\item
  Para crear una nueva línea se utiliza
  \texttt{\textbackslash{}\textbackslash{}}
\item
  Para crear puntos de alineación vertical entre varias líneas, se
  utiliza \texttt{\&}.
\end{itemize}

Generalmente los puntos de alineación vertical se establecen para queden
alineados operadores binarios como los signos \texttt{=}, \texttt{+} o
\texttt{-}. Una práctica frecuente es disponer varias ecuaciones en
varias filas de forma que los signos de igualdad \texttt{=} de las
distintas ecuaciones queden alineados verticalmente. Para obtener un
espaciado adecuado alrededor de dichos operadores, el símbolo
\texttt{\&} debe colocarse delante (y no detrás) de ellos.

\hypertarget{partir-una-ecuaciuxf3n-larga-en-varias-luxedneas-el-entorno-split}{%
\section{\texorpdfstring{Partir una ecuación larga en varias líneas: El
entorno
\texttt{split}}{Partir una ecuación larga en varias líneas: El entorno split}}\label{partir-una-ecuaciuxf3n-larga-en-varias-luxedneas-el-entorno-split}}

Veamos varias formas de partir la ecuación \[
P(x) = 1  + x + x^2 + x^3 + x^4 + x^5 + x^6 + x^7 + x^8
\] en tres líneas con el entorno \texttt{split}.

En primer lugar vamos a alinear el primer signo \texttt{+} de cada fila.
El código

\begin{Shaded}
\begin{Highlighting}[]
\SpecialStringTok{$$}
\KeywordTok{\textbackslash{}begin}\NormalTok{\{}\ExtensionTok{split}\NormalTok{\}}
\SpecialStringTok{P(x) = 1 \& + x + x\^{}2 }\SpecialCharTok{\textbackslash{}\textbackslash{}}
\SpecialStringTok{         \& + x\^{}3 + x\^{}4 + x\^{}5 }\SpecialCharTok{\textbackslash{}\textbackslash{}}
\SpecialStringTok{         \& + x\^{}6 + x\^{}7 + x\^{}8}
\KeywordTok{\textbackslash{}end}\NormalTok{\{}\ExtensionTok{split}\NormalTok{\}}
\SpecialStringTok{$$}
\end{Highlighting}
\end{Shaded}

produce el resultado \[
\begin{split}
P(x) = 1 & + x + x^2 \\
         & + x^3 + x^4 + x^5 \\
         & + x^6 + x^7 + x^8
\end{split}
\]

\begin{tcolorbox}[enhanced jigsaw, arc=.35mm, opacityback=0, bottomrule=.15mm, leftrule=.75mm, breakable, rightrule=.15mm, left=2mm, colback=white, toprule=.15mm, colframe=quarto-callout-note-color-frame]
\begin{minipage}[t]{5.5mm}
\textcolor{quarto-callout-note-color}{\faInfo}
\end{minipage}%
\begin{minipage}[t]{\textwidth - 5.5mm}

En el código de este primer ejemplo, y en todos los ejemplos
posteriores, se alinean los símbolos \texttt{\&} para que resulte más
legible, pero no es imprescindible hacerlo.

\end{minipage}%
\end{tcolorbox}

Ahora tomamos como referencia el signo \texttt{=} de la primera
ecuación, y añadimos un determinado espacio adicional al inicio de las
filas segunda y tercera (\texttt{1em} es más o menos la anchura de una
letra \texttt{m}):

\begin{Shaded}
\begin{Highlighting}[]
\SpecialStringTok{$$}
\KeywordTok{\textbackslash{}begin}\NormalTok{\{}\ExtensionTok{split}\NormalTok{\}}
\SpecialStringTok{P(x) \& = 1  + x + x\^{}2 }\SpecialCharTok{\textbackslash{}\textbackslash{}}
\SpecialStringTok{     \& }\SpecialCharTok{\textbackslash{}hspace}\SpecialStringTok{\{1em\} + x\^{}3 + x\^{}4 + x\^{}5 }\SpecialCharTok{\textbackslash{}\textbackslash{}}
\SpecialStringTok{     \& }\SpecialCharTok{\textbackslash{}hspace}\SpecialStringTok{\{2em\} + x\^{}6 + x\^{}7 + x\^{}8}
\KeywordTok{\textbackslash{}end}\NormalTok{\{}\ExtensionTok{split}\NormalTok{\}}
\SpecialStringTok{$$}
\end{Highlighting}
\end{Shaded}

El resultado es \[
\begin{split}
P(x) & = 1  + x + x^2 \\
     & \hspace{1em} + x^3 + x^4 + x^5 \\
     & \hspace{2em} + x^6 + x^7 + x^8
\end{split}
\]

Al principio se advirtió que es mejor situar el símbolo separador
\texttt{\&} antes de un operador binario, que después. Veamos que en
efecto, si colocamos el símbolo \texttt{\&} después del signo
\texttt{=}, se estropea el espacio alrededor del \texttt{=}. Escribiendo

\begin{Shaded}
\begin{Highlighting}[]
\SpecialStringTok{$$}
\KeywordTok{\textbackslash{}begin}\NormalTok{\{}\ExtensionTok{split}\NormalTok{\}}
\SpecialStringTok{P(x) = \& 1  + x + x\^{}2 }\SpecialCharTok{\textbackslash{}\textbackslash{}}
\SpecialStringTok{       \& + x\^{}3 + x\^{}4 + x\^{}5 }\SpecialCharTok{\textbackslash{}\textbackslash{}}
\SpecialStringTok{       \& + x\^{}6 + x\^{}7 + x\^{}8}
\KeywordTok{\textbackslash{}end}\NormalTok{\{}\ExtensionTok{split}\NormalTok{\}}
\SpecialStringTok{$$}
\end{Highlighting}
\end{Shaded}

resulta \[
\begin{split}
P(x) = & 1  + x + x^2 \\
     & + x^3 + x^4 + x^5 \\
     & + x^6 + x^7 + x^8
\end{split}
\]

El problema con el espaciado aparece porque no hay ningún carácter
después del signo \texttt{=}. Una forma de \emph{engañar} a LaTeX es
escribir \texttt{\{\}} después del signo \texttt{=}, que en realidad no
imprime nada, pero restablece el espaciado natural alrededor del
\texttt{=}. Escribiendo

\begin{Shaded}
\begin{Highlighting}[]
\SpecialStringTok{$$}
\KeywordTok{\textbackslash{}begin}\NormalTok{\{}\ExtensionTok{split}\NormalTok{\}}
\SpecialStringTok{P(x) = \{\} \& 1  + x + x\^{}2 }\SpecialCharTok{\textbackslash{}\textbackslash{}}
\SpecialStringTok{          \& + x\^{}3 + x\^{}4 + x\^{}5 }\SpecialCharTok{\textbackslash{}\textbackslash{}}
\SpecialStringTok{          \& + x\^{}6 + x\^{}7 + x\^{}8}
\KeywordTok{\textbackslash{}end}\NormalTok{\{}\ExtensionTok{split}\NormalTok{\}}
\SpecialStringTok{$$}
\end{Highlighting}
\end{Shaded}

obtenemos \[
\begin{split}
P(x) = {} & 1  + x + x^2 \\
     & + x^3 + x^4 + x^5 \\
     & + x^6 + x^7 + x^8
\end{split}
\]

Un bloque creado con \texttt{split} puede numerarse como si tuviera una
sola línea. Y ser referenciado posteriormente. El código

\begin{Shaded}
\begin{Highlighting}[]
\SpecialStringTok{$$}
\KeywordTok{\textbackslash{}begin}\NormalTok{\{}\ExtensionTok{split}\NormalTok{\}}
\SpecialStringTok{P(x) \& = 1  + x + x\^{}2 }\SpecialCharTok{\textbackslash{}\textbackslash{}}
\SpecialStringTok{     \& + x\^{}3 + x\^{}4 + x\^{}5 }\SpecialCharTok{\textbackslash{}\textbackslash{}}
\SpecialStringTok{     \& + x\^{}6 + x\^{}7 + x\^{}8}
\KeywordTok{\textbackslash{}end}\NormalTok{\{}\ExtensionTok{split}\NormalTok{\}}
\SpecialStringTok{$$}\NormalTok{\{\#eq{-}split\}}

\NormalTok{... la @eq{-}split ...}
\end{Highlighting}
\end{Shaded}

genera la siguiente salida:
\begin{equation}\protect\hypertarget{eq-split}{}{
\begin{split}
P(x) & = 1  + x + x^2 \\
 & + x^3 + x^4 + x^5 \\
 & + x^6 + x^7 + x^8
\end{split}
}\label{eq-split}\end{equation}

\ldots{} la Ecuación~\ref{eq-split} \ldots{}

\hypertarget{varias-ecuaciones-centradas-entorno-gather}{%
\section{\texorpdfstring{Varias ecuaciones centradas: Entorno
\texttt{gather}}{Varias ecuaciones centradas: Entorno gather}}\label{varias-ecuaciones-centradas-entorno-gather}}

El entorno \texttt{gather} agrupa varias ecuaciones, quedando cada una
centrada en su fila. Por ejemplo:

\begin{Shaded}
\begin{Highlighting}[]
\KeywordTok{\textbackslash{}begin}\NormalTok{\{}\ExtensionTok{gathered}\NormalTok{\} }
\NormalTok{2x + 3y =  10 }\FunctionTok{\textbackslash{}\textbackslash{}} 
\NormalTok{x {-} y =  0}
\KeywordTok{\textbackslash{}end}\NormalTok{\{}\ExtensionTok{gathered}\NormalTok{\}}
\end{Highlighting}
\end{Shaded}

da como resultado \[
\begin{gathered} 
2x + 3y =  10 \\ 
x - y =  0
\end{gathered}
\]

El bloque creado puede numerarse como una única ecuación. Escribiendo

\begin{Shaded}
\begin{Highlighting}[]
\SpecialStringTok{$$}
\KeywordTok{\textbackslash{}begin}\NormalTok{\{}\ExtensionTok{gathered}\NormalTok{\}}\SpecialStringTok{ }
\SpecialStringTok{2x + 3y =  10 }\SpecialCharTok{\textbackslash{}\textbackslash{}}\SpecialStringTok{ }
\SpecialStringTok{x {-} y =  0}
\KeywordTok{\textbackslash{}end}\NormalTok{\{}\ExtensionTok{gathered}\NormalTok{\}}
\SpecialStringTok{$$}\NormalTok{\{\#eq{-}gat\}}

\NormalTok{La @eq{-}gat ... El sistema [{-}@eq{-}gat] ...}
\end{Highlighting}
\end{Shaded}

obtenemos \begin{equation}\protect\hypertarget{eq-gat}{}{
\begin{gathered} 
2x + 3y =  10 \\ 
x - y =  0
\end{gathered}
}\label{eq-gat}\end{equation}

La Ecuación~\ref{eq-gat} \ldots{} El sistema \ref{eq-gat} \ldots{}

Notar que la sintaxis \texttt{{[}-@eq-gat{]}} para la segunda referencia
cruzada omite el prefijo automático ``Ecuación'' e imprime sólo el
número.

\hypertarget{varias-ecuaciones-con-alineaciuxf3n-vertical-los-entornos-aligned-y-alignedat.}{%
\section{\texorpdfstring{Varias ecuaciones con alineación vertical: Los
entornos \texttt{aligned} y
\texttt{alignedat}.}{Varias ecuaciones con alineación vertical: Los entornos aligned y alignedat.}}\label{varias-ecuaciones-con-alineaciuxf3n-vertical-los-entornos-aligned-y-alignedat.}}

Mientras que el entorno \texttt{gather} que acabamos de estudiar en la
sección previa, centra cada ecuación en su fila, los entornos
\texttt{aligned} y \texttt{alignedat} hacen uso del separador
\texttt{\&} para controlar la alineación vertical de las diferentes
ecuaciones. En esta sección veremos como utilizar los entornos
\texttt{aligned} y \texttt{alignedat} para mejorar el aspecto del
sistema \ref{eq-gat}, alineando primero el signo \texttt{=} de las dos
ecuaciones, y perfeccionando después la alineación de los signos
\texttt{+} y \texttt{-} y las incógnitas. El objetivo es conseguir que
quede así:

\[
\begin{alignedat}{2} 
2x & + {}   &  3y  & =  10 \\ 
 x & - {}   &   y  & =  0 
\end{alignedat}
\]

\hypertarget{el-entorno-aligned}{%
\subsection{\texorpdfstring{El entorno
\texttt{aligned}}{El entorno aligned}}\label{el-entorno-aligned}}

En el entorno \texttt{aligned}, el símbolo \texttt{\&} crea un punto de
alineación vertical, creando una columna alineada a la derecha a su
izquierda y una columna alineada a la izquierda a su derecha, esto es,
una disposición similar a la creada con un grupo \texttt{rl} en el
entorno \texttt{array}.

\begin{Shaded}
\begin{Highlighting}[]
\KeywordTok{\textbackslash{}begin}\NormalTok{\{}\ExtensionTok{aligned}\NormalTok{\} }
\NormalTok{2x + 3y \& =  10 }\FunctionTok{\textbackslash{}\textbackslash{}}
\NormalTok{  x {-} y \& =  0 }
\KeywordTok{\textbackslash{}end}\NormalTok{\{}\ExtensionTok{aligned}\NormalTok{\}}
\end{Highlighting}
\end{Shaded}

da como resultado \[
\begin{aligned} 
2x + 3y & =  10 \\
  x - y & =  0 
\end{aligned}
\]

El mismo ejemplo con una llave a la izquierda, con la notación habitual
para sistemas de ecuaciones:

\begin{Shaded}
\begin{Highlighting}[]
\SpecialStringTok{$$}
\SpecialCharTok{\textbackslash{}left\textbackslash{}\{}
\KeywordTok{\textbackslash{}begin}\NormalTok{\{}\ExtensionTok{aligned}\NormalTok{\}}\SpecialStringTok{ }
\SpecialStringTok{2x + 3y \& =  10 }\SpecialCharTok{\textbackslash{}\textbackslash{}}
\SpecialStringTok{  x {-} y \& =  0 }
\KeywordTok{\textbackslash{}end}\NormalTok{\{}\ExtensionTok{aligned}\NormalTok{\}}
\SpecialCharTok{\textbackslash{}right}\SpecialStringTok{.}
\SpecialStringTok{$$}
\end{Highlighting}
\end{Shaded}

\[
\left\{
\begin{aligned} 
2x + 3y & =  10 \\
  x - y & =  0 
\end{aligned}
\right.
\]

Como ya hemos comentando varias veces, para conseguir un espaciado
cuidado alrededor de operadores binarios como \texttt{=}, \texttt{+} o
\texttt{-}, el símbolo separador \texttt{\&} debe ir delante de dichos
operadores. Comparar el primer ejemplo del apartado con el siguiente,
donde el separador \texttt{\&} se coloca detrás, en vez de delante, del
operador \texttt{=}:

\begin{Shaded}
\begin{Highlighting}[]
\SpecialStringTok{$$}
\KeywordTok{\textbackslash{}begin}\NormalTok{\{}\ExtensionTok{aligned}\NormalTok{\}}\SpecialStringTok{ }
\SpecialStringTok{2x + 3y  = \& 10 }\SpecialCharTok{\textbackslash{}\textbackslash{}}
\SpecialStringTok{  x {-} y  = \& 0 }
\KeywordTok{\textbackslash{}end}\NormalTok{\{}\ExtensionTok{aligned}\NormalTok{\}}
\SpecialStringTok{$$}
\end{Highlighting}
\end{Shaded}

\[
\begin{aligned} 
2x + 3y  = & 10 \\
  x - y  = & 0 
\end{aligned}
\] Hemos perdido el espaciado natural a la derecha del signo \texttt{=}.
Esto ya nos ocurrió antes en un ejemplo del entorno \texttt{split},
sabemos que pasa por no haber ningún carácter después del signo
\texttt{=}. Lo corregimos escribiendo \texttt{\{\}} para engañar a
LaTeX, como hicimos con el ejemplo del entorno \texttt{split}.

\begin{Shaded}
\begin{Highlighting}[]
\SpecialStringTok{$$}
\KeywordTok{\textbackslash{}begin}\NormalTok{\{}\ExtensionTok{aligned}\NormalTok{\}}\SpecialStringTok{ }
\SpecialStringTok{2x + 3y  = \{\} \& 10 }\SpecialCharTok{\textbackslash{}\textbackslash{}}
\SpecialStringTok{  x {-} y  = \{\} \& 0 }
\KeywordTok{\textbackslash{}end}\NormalTok{\{}\ExtensionTok{aligned}\NormalTok{\}}
\SpecialStringTok{$$}
\end{Highlighting}
\end{Shaded}

\[
\begin{aligned} 
2x + 3y  = {} & 10 \\
  x - y  = {} & 0 
\end{aligned}
\]

Si se quieren crean varios grupos de columnas \texttt{rl}, se usa un
símbolo \texttt{\&} adicional para separar los diferentes grupos. El
código

\begin{Shaded}
\begin{Highlighting}[]
\SpecialStringTok{$$}
\KeywordTok{\textbackslash{}begin}\NormalTok{\{}\ExtensionTok{aligned}\NormalTok{\}}\SpecialStringTok{ }
\SpecialStringTok{2x + 3y \& = 10   \&    x {-} y \& = 0 }\SpecialCharTok{\textbackslash{}\textbackslash{}}
\SpecialStringTok{  x {-} y \& = 0    \&   2x + y \& = 3}
\KeywordTok{\textbackslash{}end}\NormalTok{\{}\ExtensionTok{aligned}\NormalTok{\}}
\SpecialStringTok{$$}
\end{Highlighting}
\end{Shaded}

da como resultado \[
\begin{aligned} 
2x + 3y & = 10   &    x - y & = 0 \\
  x - y & = 0    &   2x + y & = 3
\end{aligned}
\]

\hypertarget{el-entorno-alignedat}{%
\subsection{\texorpdfstring{El entorno
\texttt{alignedat}}{El entorno alignedat}}\label{el-entorno-alignedat}}

Acabamos de ver que, cuando se crean varios grupos de columnas
\texttt{rl} con el entorno \texttt{aligned}, se crea de forma automática
un espaciado horizontal predeterminado entre los grupos. Con el entorno
\texttt{alignedat} podemos controlar ese espaciado entre los grupos.

Repetimos el ejemplo anterior sin espaciado entre los dos grupos:

\begin{Shaded}
\begin{Highlighting}[]
\SpecialStringTok{$$}
\KeywordTok{\textbackslash{}begin}\NormalTok{\{}\ExtensionTok{alignedat}\NormalTok{\}}\SpecialStringTok{\{2\} }
\SpecialStringTok{2x + 3y  \& = 10   \&    x {-} y \& = 0 }\SpecialCharTok{\textbackslash{}\textbackslash{}}
\SpecialStringTok{  x {-} y  \& = 0    \&   2x + y \& = 3}
\KeywordTok{\textbackslash{}end}\NormalTok{\{}\ExtensionTok{alignedat}\NormalTok{\}}
\SpecialStringTok{$$}
\end{Highlighting}
\end{Shaded}

\[
\begin{alignedat}{2} 
2x + 3y  & = 10   &    x - y & = 0 \\
  x - y  & = 0    &   2x + y & = 3
\end{alignedat}
\]

Y ahora con un espaciado igual a \texttt{2em}:

\begin{Shaded}
\begin{Highlighting}[]
\SpecialStringTok{$$}
\KeywordTok{\textbackslash{}begin}\NormalTok{\{}\ExtensionTok{alignedat}\NormalTok{\}}\SpecialStringTok{\{2\} }
\SpecialStringTok{2x + 3y  = \& 10   \&  }\SpecialCharTok{\textbackslash{}hspace}\SpecialStringTok{\{2em\}   x {-} y \& = 0 }\SpecialCharTok{\textbackslash{}\textbackslash{}}
\SpecialStringTok{  x {-} y  = \& 0    \&   2x + y \& = 3}
\KeywordTok{\textbackslash{}end}\NormalTok{\{}\ExtensionTok{alignedat}\NormalTok{\}}
\SpecialStringTok{$$}
\end{Highlighting}
\end{Shaded}

\[
\begin{alignedat}{2} 
2x + 3y  = & 10   &  \hspace{2em}   x - y & = 0 \\
  x - y  = & 0    &   2x + y & = 3
\end{alignedat}
\]

Notar que el entorno \texttt{alignedat} necesita un argumento, que se
indica entre llaves, y que en nuestros ejemplos tiene el valor
\texttt{2} (\texttt{\textbackslash{}begin\{alignedat\}\{2\}}). Este
argumento indica el número de grupos de columnas \texttt{rl} (o,
equivalentemente, el número de símbolos \texttt{\&} más \(1\), dividido
entre \(2\)).

\hypertarget{sistemas-de-ecuaciones-lineales-con-el-entorno-alignedat}{%
\subsection{\texorpdfstring{Sistemas de ecuaciones lineales con el
entorno
\texttt{alignedat}}{Sistemas de ecuaciones lineales con el entorno alignedat}}\label{sistemas-de-ecuaciones-lineales-con-el-entorno-alignedat}}

El control del espaciado entre grupos de columnas \texttt{rl} con
\texttt{alignedat} puede (ab)usarse para mejorar la alineación de los
signos e incógnitas en un sistema de ecuaciones lineales. Volvemos a
escribir el sistema al inicio de la sección usando el entorno
\texttt{alignedat} con dos grupos de columnas \texttt{rl}:

\begin{Shaded}
\begin{Highlighting}[]
\SpecialStringTok{$$}
\KeywordTok{\textbackslash{}begin}\NormalTok{\{}\ExtensionTok{alignedat}\NormalTok{\}}\SpecialStringTok{\{2\} }
\SpecialStringTok{2x \& +    \& 3y \& =  10 }\SpecialCharTok{\textbackslash{}\textbackslash{}}\SpecialStringTok{ }
\SpecialStringTok{ x \& {-}    \&  y \& =  0 }
\KeywordTok{\textbackslash{}end}\NormalTok{\{}\ExtensionTok{alignedat}\NormalTok{\}}
\SpecialStringTok{$$}
\end{Highlighting}
\end{Shaded}

El resultado es \[
\begin{alignedat}{2} 
2x & +    & 3y & =  10 \\ 
 x & -    &  y & =  0 
\end{alignedat}
\]

Hemos conseguido que los signos \texttt{+} y \texttt{-} y las incógnitas
\texttt{x} e \texttt{y} queden alineados verticalmente. Pero se aprecia
que no hay espacio alrededor de los signos \texttt{+} y \texttt{-} (como
sí lo hay alrededor del signo \texttt{=}). Esto ya nos ocurrió antes, en
un ejemplo del entorno \texttt{split} y otro del entorno
\texttt{aligned}. Sabemos que pasa por no haber ningún carácter después
de los signos \texttt{+} y \texttt{-}, y que podemos corregirlo
escribiendo \texttt{\{\}} después de dichos signos. Aprovechamos para
añadir una llave delimitando el sistema, ahora a la derecha.

\begin{Shaded}
\begin{Highlighting}[]
\SpecialStringTok{$$}
\SpecialCharTok{\textbackslash{}left}\SpecialStringTok{.}
\KeywordTok{\textbackslash{}begin}\NormalTok{\{}\ExtensionTok{alignedat}\NormalTok{\}}\SpecialStringTok{\{2\} }
\SpecialStringTok{2x \& + \{\}   \&  3y  \& =  10 }\SpecialCharTok{\textbackslash{}\textbackslash{}}\SpecialStringTok{ }
\SpecialStringTok{ x \& {-} \{\}   \&   y  \& =  0 }
\KeywordTok{\textbackslash{}end}\NormalTok{\{}\ExtensionTok{alignedat}\NormalTok{\}}
\SpecialCharTok{\textbackslash{}right\textbackslash{}\}}
\SpecialStringTok{$$}
\end{Highlighting}
\end{Shaded}

\[
\left.
\begin{alignedat}{2} 
2x & + {}   &  3y  & =  10 \\ 
 x & - {}   &   y  & =  0 
\end{alignedat}
\right\}
\]



\end{document}
